\documentclass[11pt]{myclass}

\usepackage{amssymb,amsmath,amsfonts,pdfpages,color, url}
\usepackage{booktabs}
\usepackage{wrapfig}
\usepackage{amsmath}
\usepackage{multirow}

\newtheorem{cor}[theorem]{Corollary}
\newtheorem{rem}{Remark}[section]
\newtheorem{addendum}[theorem]{Addendum}
\newtheorem{definition}[theorem]{Definition}
\newtheorem{exa}{Example}[section]
\newtheorem{Notation}[theorem]{Notation}
\newtheorem{question}[theorem]{Question}
\newtheorem{convention}[theorem]{Convention}
\newtheorem{Assumption}[theorem]{Assumption}
\newcommand{\R}{\ensuremath{\mathbb{R}}}

\def\dis{\mathop{\displaystyle}}
\def\Train{\mathop{\rm Train}}
\def\Test{\mathop{\rm Test}}
\newcommand{\bb}{\ensuremath{{\rm{\bf b}}}}
\newcommand{\w}{\ensuremath{{\rm{\bf w}}}}
\def\Gsim{\mathop{\Gamma}}

%%% ----------------------------------------------------------------------


\begin{document}


\title{Project Intermediate Report \\ Title: Inferring Transportation modes from GPS Trajectories}

\author{Sole Team Member: Hasan Pourmahmoodaghababa \\ %University of Utah \\ 
\texttt{uID: u1255635}}




\maketitle

\section{Progress}

I downloaded the data from 

\url{https://www.microsoft.com/en-us/research/publication/}

\hspace{1.3cm} \url{gps-trajectories-with-transportation-mode-labels/}, \\
but, unfortunately, I noticed that it is not complete. However, I found the complete one from 

\url{https://msropendata.com/datasets/d19b353b-7483-4db7-a828-b130f6d1f035}. \\
In this dataset, which includes the previous data, there are 69 users (it was only 24 users in the previous data) with transportation modes labels. 

I have done the preprocessing steps up to now. Indeed, I tried to match the label dataset of each user to its corresponding trajectories, but I found some of the labels cannot be matched with any part of any trajectory from that user as well as some parts of some trajectories cannot be matched with any label. 

Interestingly, I emailed some people who have worked on this dataset including Dr. Yu Zheng from Microsoft Research Asia, who is addressed in the user's guide of data to be contacted, to see if they had already encountered this problem and what they have done. However, no one replied me! Therefore, I decided to do as follows: 

\begin{enumerate}
\item There are 12 labels (walk, bike, bus, car, taxi, subway, railway, train, airplane, motorcycle, run, other). As it is recommended in the user's guide of the data, we can combine some of labels like car and taxi, train, railway and subway. After these sorts of merges, we ended up with 5 major transportation modes (walk, bike, bus, car, train) which constitute the majority of data and are used in other studies (see \cite{DH2018} for example). 

\item  I assigned matchable labels to their corresponding trajectories. 

\item I removed all trajectories without a label.

\item I removed any part of any trajectory without a label.

\item I partitioned each user's GPS trajectory into some sub-trajectories (trips) according to the time interval. In fact, I used the threshold 20 minutes recommended in \cite{ZLWX}. 

\item This way I obtained about 10,000 trajectories with only 1 label, 1,700 trajectories with 2 labels and less than 100 trajectories with 3 or 4 labels. 

\item My idea is to use each part of a trajectory that is labeled the same for a number of consecutive segments as a word. Nonetheless, at this point, I have almost all of my sentences with length 1 or 2. Therefore, I have decided to concatenate some trajectories from 10,000 sole labeled trajectories in order to make some sentences with length 3, 4 and 5. 

\item To have a roughly balanced data, I will use only a fraction of length 1 sentences but all length 2 sentences. 

\item Finally, I think, I will have about 6,000 trajectories, i.e., 6,000 sentences with length varying from 1 to 5. Hence, I can do something like the P-O-S tagging task on this dataset in order to predict the transportation modes of a trajectory given its spans. 

\item I have defined a feature mapping for trajectories that assigns their average length, average uniform velocity and average uniform acceleration. 
\end{enumerate}

My idea is to use these 3 features (possibly I will add other features like the endpoints of the trajectory) to map a trajectory (i.e. a word in a sentence) to $\mathbb{R}^3$. Then I will be able to use the previous label $y_{i-1}$ along with the mapped trajectory to predict the label of next state (i.e. word/sub-trajectory). So, the model can be $p(y_i | y_{i-1}, \phi(x_i))$ or $p(y_i | y_{i-1}, \phi(x))$  that can be done with a CMM for instance. 

\bibliographystyle{plain}
\bibliography{references}

\end{document}